\documentclass{article}
% html and hyperref both define \hyperref, but hyperref's seem to be better,
% so it goes second.
\usepackage{html} % LaTeX2HTML utilities
\usepackage{hyperref}

\newcommand{\baseurl}{\latexhtml{http://www.mit.edu/~tech-squares/govdocs/}{}}
\newcommand{\matchextension}{.\latexhtml{pdf}{html}}
\newcommand{\matchlink}[2]{\href{\baseurl#1\matchextension}{#2}}

\title{Photo Policy}
\latex{\author{Tech Squares}}
\date{}
\hypersetup{pdftitle=Tech Squares Photo Policy}

% Margins
\oddsidemargin=.125in
\textwidth=6.25in
\topmargin=0in
\headheight=0in
\headsep=0in
\textheight=9in
\parskip=.4\baselineskip % blank space between paragraphs

\begin{document}

\maketitle

Tech Squares maintains a photo gallery and uses some photos for publicity purposes. (For the purposes of this policy, we include videos as photos.) Participants who do not wish to appear in photographs may inform the \href{http://www.mit.edu/~tech-squares/officers.html}{Club Photographer or Safer Dances Coordinators} (contact squares AT mit.edu or in person), who jointly maintain a list.

Members of the Tech Squares community can create accounts on our photo gallery to view member-only photos and upload photos. Uploaded photos will held in a queue until they are checked against the “no-photographs” list; photos which our moderator recognizes as containing such participants will be rejected. Additionally, we will remove any photo from the gallery upon request from someone in the photo.

Photos taken in 2018 or later that are added to the gallery will be visible to the public. Photos taken in 2017 or earlier will default to be visible only to logged-in members.

Tech Squares will acquire permission from the photographer and any easily identifiable subjects of a photo before using it in publicity (such as posters, club-created social media posts, or prominently on our website). Additionally, we will remove any photo from our publicity upon request from someone in the photo, including making a good-faith effort to remove posters with photos that we put up around MIT.

Photographers should know that some of our members prefer not to appear in photographs, and that harassing photography and similar behavior is subject to our \matchlink{safer-dances}{Safer Dances Policy}. If you plan on taking photos or videos at a Tech Squares event, we encourage you to let others know you are doing so, so that anyone who objects can let you know, and/or to check with an officer to see if anyone on the no-photographs list is present.

\end{document}
