\documentclass{bylaws}
\title{Tech Squares Constitution}
\date{28 April 2015}
\begin{document}
\maketitle

\LARGE{Note for ASA/SOLE: This is a draft constitution that has not yet been approved by the club, because we have historically required long notice to make updates. We anticipate that it will be adopted this year.}

\begin{history}
Adopted 22 May 1984
Amended 19 January 1988
Amended 1 March 1988
Amended 16 August 1994
Amended 13 August 2002
Amended 23 April 2013
Amended 3 September 2013
Rewritten 28 April 2015
Amended 29 November 2016
Amended 25 October 2022
Amended ?? December 2025
\end{history}


\article{Name and Purpose}
\section{}This organization shall be called Tech Squares, the Square and Round Dance Club of the Massachusetts Institute of Technology, hereinafter referred to as \textit{the club} and \textit{MIT}, respectively.
\section{}The club shall regularly hold Modern Western Square Dances.
\section{}The club shall hold classes to introduce beginners to square dancing.


\article{Membership}
\section{}Upon graduation from the class, a person becomes a \textit{club dancer}.
\section{}A person who has not graduated from the class must dance club level to become a club dancer. Other requirements may be stated in the standing policies.
\section{}Club dancer status shall include the right to wear the club badge. No one but club dancers may wear the club badge.
\section{}Any club dancer who has attended at least four recent weekly dances shall be a \textit{voting dancer}. For this purpose, a recent weekly dance shall be any since the fifteenth most recent non-summer (not including June, July, and August) weekly dance, inclusive.
\section{}Only voting dancers may vote at club meetings or join the Executive Committee.
\section{}The Executive Committee may revoke club dancer status by two-thirds majority of those voting.


\article{Dances}
\section{}The club shall choose a \textit{club level} for its square dances.
\section{}The club shall hold \textit{weekly dances} year-round allowing exceptions for holidays, weather, and similar occasions. Weekly dances shall be at club level, except as required for the class.
\section{}The club may hold additional dances and workshops at club level or other levels.


\article{Class}
\section{}The club shall hold a class during each Fall and Spring academic term at MIT to introduce beginners to square dancing. The class shall teach club level.
\section{}At least one class per year shall be open to the general public.
\section{}The class shall be coordinated by the Class Coordinator.
\section{}The club caller shall be responsible for teaching the class.
\section{}The standing policies shall state criteria for graduation from the class.


\article{Employees}
\section{}The \textit{club caller(s)} shall be defined as the employee(s) who are hired to be primarily responsible for calling square dances at weekly dances. The \textit{club cuer(s)} shall be defined as the employee(s) who are hired to be primarily responsible for cueing round dances at weekly dances.
\section{}No person may be an Executive Committee member and club caller or cuer concurrently.
\section{}The club caller and the club cuer shall be ex officio club dancers, but not ex officio voting dancers, for the term of their employment. They may wear the club badge. They may attend club or Executive Committee meetings only at the invitation of the Executive Committee.
\section{}The selection of a club caller or club cuer must be approved by the club.
\section{}The Executive Committee may hire other callers and cuers.  Such employees do not become club dancers by fact of their employment by the club.


\article{Club Meetings}
\section{}The club shall regularly hold meetings, called \textit{club meetings}, to review the club status and conduct club business.
\section{}Club meetings shall be open to all club dancers except as otherwise noted in the constitution or standing policies.
\section{}Club meetings shall be run by the President by default, and otherwise those present shall select a chair.
\section{}A quorum shall consist of one-quarter of the total voting dancers.
\section{}At club meetings, no business other than reports and discussions shall be carried on unless a quorum is present.
\section{}Abstentions count toward quorum, but do not count as voting.
\section{}If fewer than 50\% of those voting are MIT students, then the votes of the MIT students shall be scaled so as to collectively constitute 50\% of the resulting total voting power.
\section{}A ``two-thirds vote'' refers to at least two-thirds of the voting power of those present and voting being cast in favor. Similar language applies with other fractions. A majority vote shall mean more than half the voting power of those present and voting being cast in favor.
\section{}Decisions of any club meeting shall be binding upon the Executive Committee and the club coordinators.


\article{Executive Committee}

\section{}The club shall have an \textit{Executive Committee} (EC). The EC shall be responsible for the affairs of the club and is empowered to make such decisions as it feels are necessary to efficiently carry out the business of the club, except as otherwise governed by this constitution or the standing policies.
\section{}The members of the Tech Squares Executive Committee shall include the four key officers: President, Treasurer, Class Coordinator, and Publicity Coordinator.

\subarticle{Responsibilities}
\subsection{}The President shall be responsible for the overall operation of the club.
\subsection{}The Treasurer shall be responsible for the finances of the club.
\subsection{}The Class Coordinator shall be responsible for the organization of the class.
\subsection{}The Publicity Coordinator shall be responsible for organizing all class and club publicity.

\subarticle{Deputies}
\subsection{}Each of the four officers may, at time of election or at any later point, request a deputy. The Executive Committee may also request a deputy for any of the four officers at any time by majority vote.
\subsection{}If a deputy is requested during the election, the position will be filled as usual in that election. If a deputy is requested at any other time, the position is filled as with any other vacancy on the Executive Committee.
\subsection{}The EC shall consist of the four key officers and (as applicable) the deputies to each of those officers (combined, the elected coordinators), for a total of four to eight members.
\subsection{}Each EC member (key officer or deputy) must be a different person.

\subarticle{Operations}
\subsection{}A majority vote of the Executive Committee shall be necessary for any official decisions it makes.
\subsection{}The Executive Committee may appoint and remove \textit{appointed coordinators} to assist with running the club, as it feels appropriate or as dictated by the standing policies.
\subsection{}The Executive Committee may overrule any coordinator, but shall exercise restraint in doing so.


\article{Elections and Vacancies}

\subarticle{Normal Elections}
\subsection{}Elections of Executive Committee members shall be held each year in the spring. Executive Committee members shall take office on the earlier of five full calendar days after the spring class ends or June 1. Elections shall be held four to eight weeks prior to the expected start of their term of office.
\subsection{}An Executive Committee member may serve until their successor's term begins in the next spring (about one year), or they may choose to vacate their office at the end of fall term. Fall elections shall be held for positions that are being vacated. Executive Committee members elected in the fall take office on the earlier of five full calendar days after the fall class ends or January 1.
\subsection{}Executive Committee members must be voting dancers at the time of elections and have been club dancers for at least three months. The latter requirement may be waived by a two-thirds vote of the club. Advance notice is not required.
\subsection{}Elections must occur at a club meeting.

\subarticle{Vacancies}
\subsection{}If there is a vacancy in the office of President, the Executive Committee shall appoint a temporary President by majority vote.
\subsection{}In the case of a vacancy in another Executive Committee position, the President appoints a temporary officer or deputy. Such temporary officer or deputy may exercise all powers of the officer or deputy they replace, except that they shall not be an EC member or be able to vote with the EC. Such temporary officers or deputies need not satisfy the normal eligibility or EC composition requirements.
\subsection{}In all cases, an election is to be held within four weeks to appoint a replacement who will serve the remainder of the original term of office.

\subarticle{Removals}
\subsection{}Upon petition by at least 20\% of the voting dancers of the club or one-half the members of the Executive Committee, the Executive Committee shall call a club meeting to discuss the removal of a club coordinator.
\subsection{}The coordinator may be removed by a three-fourths vote.


\article{Amendments}
\section{}This constitution may be amended at a club meeting by a two-thirds vote.
\section{}A proposed amendment to the constitution must be submitted in writing to the Executive Committee. The EC shall publicize the proposed amendment to the club at least two weeks before it will be voted on.
\section{}Changes to the amendment can be made during those two weeks, including at the meeting, so long as the changes preserve the essential nature of the amendment.
\section{}The EC may, by majority vote, update Article \ref{art:mit-rules} to include text that is required by the MIT ASA or SOLE, or to remove text that is no longer required. The EC may include subarticle titles for organizational purposes. The EC must inform the club of any such changes promptly, but there is no required waiting period to adopt such changes.

\article{Required MIT rules}
\label{art:mit-rules}

\subarticle{Membership}
\subsection{}All MIT students are eligible for membership in the organization.
\subsection{}The organization’s membership will at all times consist of at least 10 MIT students.
\subsection{}The organization shall not discriminate based on any characteristic listed in the MIT Nondiscrimination Policy for membership, officer position, or in any other aspect.
% We have an executive committee, not an executive board, but we're not allowed to change this text.
\subsection{}The Executive Board reserves the right to revoke the membership or officer position of any member, by a two-thirds majority vote, for non-conduct related reasons. All conduct-related complaints should be referred to the appropriate channel(s) for resolution, including, without limitation, the Institute Discrimination \& Harassment Response (IDHR) Office, the Committee on Discipline (COD), and the Student Organizations, Leadership and Engagement (SOLE) Office.

\subarticle{Officers}
\subsection{}Only MIT students are eligible to serve as officers in the organization.
% Yes, Membership refers to an "Executive Board" and officers refers to "executive board". Dunno why they aren't capitalized the same.
\subsection{}The executive board of the organization must include at least one president and one treasurer (or corresponding positions) and each officer role must be held by a distinct MIT student, meaning one student cannot serve as both president and treasurer simultaneously.

\subarticle{Governance}
\subsection{}In no event shall any proposed amendment conflict with ASA guidelines, policies, or other MIT policies.
% The instructions do say "exactly as written"...
\subsection{}[Student Organization Name] agrees to abide by the policies and guidelines of the Association of Student Activities (ASA), its Executive Board,
as well as the MIT Mind and Hand Book and the MIT Student Organization Handbook, all as may be subject to change from time to time. This
Constitution, amendments to it, and the Bylaws of this organization shall be subject to review by the ASA Executive Board, to ensure that they
are in accordance with the aforementioned policies and guidelines.

\end{document}
