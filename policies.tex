\documentclass{bylaws}
\title{Tech Squares Standing Policies}
\date{25 October 2016}

\newcommand{\sptimes}[2]{\emph{Adopted #1. To be reviewed #2.}}
\renewcommand{\thesection}{Standing Policy \arabic{section}}

\usepackage{hyperref}

\begin{document}

\maketitle

\policy{Adopting and Amending Standing Policies}
\label{pol:adopt}
\sptimes{Spring 2015}{Fall 2018}
\section{}Standing policies may be adopted or amended by majority vote of the club. Proposed policies or amendments must be submitted to the club at least two weeks prior to being voted on. Changes to the amendments can be made during those two weeks, including at the meeting, so long as the changes preserve the essential nature of the amendment.
\section{}Standing policies may be adopted or amended by majority vote of the EC. The EC shall inform the club of any such changes it makes to the standing policies, and no changes shall take effect until two weeks after said notification. Upon petition by five voting members of the club, or majority vote of the EC, any EC change shall be delayed until addressed at a club meeting. Alternatively, the EC may choose to discard any change rather than bringing it to club vote.
\section{}Each standing policy shall have a date by which it will be reviewed. Unless otherwise specified, policies shall be reviewed by the end of the third fall semester after adoption or re-approval.
\section{}For each standing policy, early in the semester when it will be reviewed, the EC shall publicize and hold an open EC meeting to discuss the policy and any needed amendments. The EC shall decide whether to re-approve it (possibly with amendments), remove it, or refer it to the club for further discussion.


\policy{Club Level}
\sptimes{Spring 2015}{Fall 2020}
\section{}The club level shall be the \textsc{Callerlab} Plus program.


\policy{Membership}
\sptimes{Spring 2015}{Fall 2020}

\section{}At its discretion, the Executive Committee may grant membership to prospective members outside the class process. Such prospective members must dance club level proficiently and have attended at least four of fifteen consecutive weekly dances.
\section{}Club membership does not expire, though a club member may relinquish membership, and membership may be revoked per \ref{pol:safer}.


\policy{Elections and Appointments}
\label{pol:electappoint}
\sptimes{Spring 2015}{Fall 2020}

\subarticle{Elections}
\label{pol:elect}

\subsection{}The elected positions shall be elected in the order in which the offices are mentioned in Article VII, Section 2 of the constitution. 
\subsection{}Approval voting shall be used. 
\subsection{}For each position,
\duty Each candidate shall have a chance to state a platform and answer questions.
\duty The club shall have an opportunity to discuss in private, including for uncontested positions.
\duty The club may ask individual candidates questions in private.
\duty To be elected, a candidate must receive approval from at least a one-half vote.  
\subsection{}Candidates must accept their nomination to run. Candidates should be present for the election meeting if possible but are not required to be. Candidates are encouraged to make themselves available to answer questions if they cannot attend in person.
\subsection{}Candidates present for the election may vote by written ballot once the club has completed asking all questions, and their votes shall be tallied with the rest of the votes.
\subsection{}After electing all other candidates, Members-at-Large will be elected to fill out the EC. Member-at-Large candidates cannot already be on the EC they are being elected to in another capacity. Some Members-at-Large may need to be students. All student-only Member-at-Large positions will be elected simultaneously first, followed by all remaining Member-at-Large positions.

\subarticle{Appointments}

\subsection{}At any time, the Executive Committee may appoint additional, non-EC officers, including but not limited to any required by the constitution or standing policies.
\subsection{}The EC shall solicit suggestions from the club for each appointed position at least once each year, around the beginning of their term.
\subsection{}Within a month of taking office, the incoming EC shall select and install a slate of appointed officers. The EC is encouraged to select their appointed officers before taking office, so that the EC and appointed officers can take office simultaneously.


\policy{Meetings}
\sptimes{Spring 2015}{Fall 2020}

\subarticle{Executive Committee}

\subsection{}Executive Committee members shall attend Executive Committee meetings.
\subsection{}Executive Committee meetings shall be announced to the Executive Committee members at least 24 hours in advance.
\subsection{}Meetings shall be open to the club except when sensitive or confidential information will be discussed. Open meetings shall be announced to the club in advance.
\subsection{}Quorum shall be more than half of the EC.

\subarticle{Club Meetings}
\subsection{}Club members must be present to vote. Voting proxies and written absentee ballots are not allowed at club meetings. However, candidates present for their election may vote by written ballot, as described in \ref{pol:electappoint}.
\subsection{}Club meetings shall be announced to the club two weeks in advance. In case of cancelled dances or lack of quorum, the meeting may be rescheduled with at least five days notice.
\subsection{}Club meetings may be called by the EC, or by petition of five voting members.
\subsection{}Upon petition of five voting members, the EC shall put requested items on the agenda for the next club meeting and announce them appropriately.


\policy{Class}
\label{pol:class}
\sptimes{Spring 2015}{Fall 2020}
\section{}The class shall be open to anyone, except that the EC may, at its discretion and for up to one class per year, limit the class to MIT students only.
\section{}The Class Coordinator, after notifying the EC, may remove people from the class if they are interfering with the class or preventing other class members from learning.
\section{}The EC must approve the list of class graduates. The Class Coordinator shall provide the EC with a list of class members, and provide a recommendation of which should graduate, which should not, and which deserve discussion.
\section{}A class member must dance club level or be expected to dance club level soon to graduate.
\section{}The class coordinator and up to three class assistants they designate may receive free subscriptions for the term the class is in session. Each person receiving a free subscription must be missing at least half of the club tips due to their service on the class committee. The class coordinator must inform the EC who should receive free subscriptions.


\policy{Duties}
\sptimes{Spring 2015}{Fall 2019}

\subarticle{Elected officers}

\subsection{}The officers may delegate tasks and appoint assistants, but are responsible for ensuring that all duties are handled, either by the officers themselves or their delegates.

\subsection{}The President and Vice President shall
\duty be responsible for all club operations
\duty oversee all other officers
\duty call and chair meetings of the club and the Executive Committee
\duty be a member of all committees
\duty appoint temporary officers as needed
\duty act as a liaison with the ASA and other outside bodies

\subsection{}The Treasurer and Vice Treasurer shall
\duty be responsible for the club's financial transactions
\duty ensure that gate fees and other accounts receivable are collected and deposited
\duty ensure that caller fees and other accounts payable are disbursed
\duty keep financial records of the club
\duty report annually to the club on the financial status of the club

\subsection{}The Class Coordinator shall
\duty organize the class
\duty ensure that admission fees from class members are collected and deposited
\duty take attendance of class members
\duty maintain lists of calls and definitions taught in the class
\duty run class meetings
\duty prepare recommended lists of graduates, as detailed in \ref{pol:class}, and ensure the EC votes on them
\duty order club badges for new graduates and other club members
\duty add new graduates to the club roster and mailing lists
\duty organize graduation
\duty act as a liaison between the class and club

\subsection{}The Publicity Coordinator shall
\duty publicize the start of Tech Squares classes and other events, as appropriate
\duty create club-related publicity, as requested by appropriate officers
\duty manage the club's social media presence
\duty design and distribute posters
\duty prepare succinct announcements to be read at weekly dances as needed
\duty send weekly emails to the club informing them of the location and schedule of dances

\subsection{}The Members-at-Large have no pre-defined tasks. They are encouraged to choose specific ways to help the club during their term. They will preferably consider such ways prior to election and include their preferences in their platform.

\subarticle{Appointed officers}
\subsection{}The Booking Director shall
\duty arrange contracts with callers and cuers for Tech Squares events
\duty make arrangements for temporary callers or cuers when the club caller or cuer is unavailable
\duty confirm arrangements with the caller and cuer shortly before each event
\duty coordinate with the Treasurer to ensure payment of callers and cuers
\subsection{}The Rooming Director shall reserve rooms and handle event registration for weekly dances and other club functions, and help other club officers with rooming-related needs.
\subsection{}The Rounds Coordinators shall manage the club's round dancing program.

\subarticle{Club Jobs}
\subsection{}Club dances shall be run by the officers and other club members.

\subsection{}A schedule assigning duties to these people will be published in advance of each dance. Such duties may include
\duty opening and setting up the hall
\duty assisting the caller and cuer, including helping them find the hall
\duty coordinating refreshments
\duty taking admission and attendance of club members and guests
\duty making announcements at dances
\duty tallying money at the end of the dance, and arranging a deposit if needed
\duty ensuring the hall is closed properly after dances

\subarticle{Additional duties}
\subsection{}The EC has several additional collective responsibilities. EC members, especially Members-at-Large, should assume responsibility for some of these jobs, and the EC should appoint officers to do the other jobs.
\subsection{}These duties include
\duty \textbf{Recording secretary}: taking minutes at club and Executive Committee meetings
\duty \textbf{Corresponding secretary}: handling club correspondence
\duty \textbf{Roster manager}: maintaining a list of members and preparing lists of voting members as needed
\duty \textbf{Club photographer}: taking photos and videos of club events and adding them to a club photo gallery
\duty \textbf{Webmaster}: updating the website when other EC members are unable or do not wish to do so themselves
\duty \textbf{Saturday dance coordinator}: ensuring the smooth functioning of Saturday dances and that all Saturday dance jobs are handled

\policy{Safer Dances}
\label{pol:safer}
\sptimes{Summer 2016}{Fall 2018}

\section{} The \href{safer-dances.html}{Safer Dances Policy} and associated \href{safer-dances-procedures.html}{procedures} are standing policies of Tech Squares and incorporated by reference.
\section{} The policy and procedures may be amended and shall be reviewed as with any other standing policy, as per SP1 ``Adopting and Amending Standing Policies''.
\section{} In addition, the names of the Safer Dances Coordinators in the policy shall be updated to reflect the people the EC has currently appointed to that office. Such updates do not require additional EC or club approval, beyond that required to appoint them.

\end{document}
