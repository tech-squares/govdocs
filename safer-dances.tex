\documentclass{article}
% html and hyperref both define \hyperref, but hyperref's seem to be better,
% so it goes second.
\usepackage{html} % LaTeX2HTML utilities
\usepackage{hyperref}

\newcommand{\baseurl}{\latexhtml{http://www.mit.edu/~tech-squares/govdocs/}{}}
\newcommand{\matchextension}{.\latexhtml{pdf}{html}}
\newcommand{\matchlink}[2]{\href{\baseurl#1\matchextension}{#2}}

\title{Safer Dances Policy}
\latex{\author{Tech Squares}}
\date{}
\hypersetup{pdftitle=Tech Squares Safer Dances Policy}

% Margins
\oddsidemargin=.125in
\textwidth=6.25in
\topmargin=0in
\headheight=0in
\headsep=0in
\textheight=9in
\parskip=.4\baselineskip % blank space between paragraphs

\begin{document}

\maketitle

Tech Squares strives to have an environment that is comfortable for all participants. Square dancing involves close physical proximity, hand-holding, and other forms of physical contact, which are not present in many other activities. While at a Tech Squares event, participants should get consent for all interactions that are not part of the cultural norm at Tech Squares. We do not tolerate harassment of dancers or other Tech Squares participants in any form.

Tech Squares participants violating these rules may be sanctioned at the discretion of the Executive Committee. As an MIT student group, Tech Squares is also bound by MIT policies (such as the \href{http://handbook.mit.edu/harassment}{harassment} and \href{http://web.mit.edu/referencepubs/nondiscrimination/}{non-discrimination} policies) and applicable laws. Violations of this policy and instances of concerning behavior can be reported to a Tech Squares Safer Dances Coordinator (Alex Dehnert or Ginda Fisher; safer-squares AT mit.edu), the organizer of the event, another \href{http://www.mit.edu/\~{}tech-squares/officers.html}{Tech Squares officer} whom you feel more comfortable with or who is more immediately available, or appropriate MIT officials or law enforcement personnel (see the \hyperref[sec:resources]{Resources section}).

This policy applies to all Tech Squares events, defined as not only our regular Tuesday and Saturday dances but also all other events using Tech Squares resources, such as rooms reserved through Tech Squares. This policy also applies to interactions taking place in relation to Tech Squares events (for example, conversations after the official ending of a Tuesday dance).

\section{Cultural Norms and Prohibited Behavior}

At Tech Squares, expected behavior between fellow dancers includes physical contact (mainly on the arms, hands, and shoulders) during dancing, eye contact while dancing, and friendly conversation during breaks. Our culture includes a number of optional flourishes, and dancers should be aware that other dancers may decline them. Dancers may offer flourishes, but should only complete them if the other dancer is also offering to do the flourish. Dancers who are unsure of how to accept or decline flourishes, or how to interpret others' responses, are encouraged to ask an officer for advice or seek verbal consent from other dancers before flourishing.

At Tech Squares, most dancers switch partners between tips and dance with a wide variety of other participants. Dancers always have the right to decline to dance with anyone without a need to give a reason; a simple ``No thanks'' should always suffice. Declining to dance with one person does not preclude dancing that tip with someone else. Pressuring people to dance, or asking someone to dance when they have already indicated disinterest in dancing with you, is unacceptable.

Many of our members have developed strong relationships with each other, and display more intimate behavior (hugs, intimate conversation, etc.) but it should be stressed that hugging or other intimate behavior requires consent and is not a generally expected part of our events.

Prohibited behavior includes, but is not limited to:
\begin{itemize}
\item Inappropriate physical contact
\item Intentionally hurting others, or acting in ways likely to hurt them
\item Unwelcome sexual attention
\item Verbal comments that are demeaning, disempowering, or discomforting based on appearance or identity
\item Unwanted or unwelcome behavior (sexual or otherwise) which makes a person feel offended, humiliated, or intimidated
\item Yelling or swearing at others or using language reasonably expected to make another person feel inferior
\item Intrusive questions or statements about others' private lives
\item Stalking, deliberate intimidation, or unsolicited following
\item Harassing photography or recording
\item Ignoring reasonable requests to change one's behavior, regardless of whether that behavior is a normal part of Tech Squares culture
\item Advocating for or encouraging any of the above behavior
\end{itemize}

\section{Enforcement}
Participants asked by anyone to stop any prohibited behavior are expected to comply immediately.
If a participant engages in prohibited behavior, Tech Squares officers may take any actions necessary to keep our events a welcoming environment for the other participants. In many cases, the necessary action will be an informal conversation about how some behavior is making someone uncomfortable or a formal warning to stop a certain behavior. Officers may impose more serious sanctions when necessary. Serious long-term sanctions, such as bans from all Tech Squares events, require Executive Committee approval, and should be reserved for egregious violations of this policy or as a last resort when other actions have not stopped the behavior. In general, sanctions will be proportionate to the severity of the behavior.

Tech Squares officers (including event organizers) may take immediate action to redress anything designed to, or with the clear impact of, disrupting the event or making the environment hostile for any participants.

Further details on the enforcement process are in the \matchlink{safer-dances-procedures}{procedures document}.

\section{Reporting}

\subsection*{When to Report}

If someone makes you or anyone else feel unsafe or unwelcome, please report it as soon as possible. You can report violations of this policy as well as any behavior that you find concerning, whether or not it violates this policy.

If you hear from someone that they feel uncomfortable or unsafe with someone else's behavior, please encourage them to report it or ask their permission to report it yourself. Even if you think it isn't a major issue, we want to know about it so we can address issues early, identify patterns, and prevent situations from escalating.

If you choose to talk to someone about their behavior directly, you can still report it. Reporting it will bring it to the attention of the Safer Dances Coordinators, so they can keep an eye on the situation and can provide sanctions later if necessary.

Safer Dances Coordinators and other officers can also help you confront someone about their concerning behavior, or speak with them on your behalf. Officers can provide whatever support you need if you choose to talk to someone directly, and they can also speak to them without you, with or without mentioning your name.

Your courage in coming forward can keep incidents from being repeated. People like you make our event a better place.

\subsection*{How to Report}

You can report to Tech Squares by talking in person to a Tech Squares Safer Dances Coordinator, the organizer of the event, or another Tech Squares officer whom you feel more comfortable with or who is more immediately available. You can also email the Safer Dances Coordinators at safer-squares AT mit.edu. You can also report to appropriate MIT officials or law enforcement personnel (see the \hyperref[sec:resources]{Resources} section).

When taking a report, our event organizers and other officers will ensure you are safe and cannot be overheard before we ask you to tell us about what happened. This can be upsetting, but we'll handle it as respectfully as possible, and you can bring someone to support you. You won't be asked to confront anyone and we won't tell anyone who you are without your permission. If you have concerns about how the process works or who will be involved, we're happy to talk about your concerns and try to alleviate them.

If there are any specific actions you want the Safer Dances Coordinators or other officers to take, you can let us know. For example, you might be reporting a possible pattern and want it added to our records; wish for somebody to be warned; or desire more serious sanctions like a ban. We don't promise to apply the sanctions you suggest, but we will consider them, and they may help get everyone on the same page in terms of severity.

Reports made to event organizers and Tech Squares officers will generally be passed along to the Safer Dances Coordinators. The Safer Dances Coordinators will keep your information private, but may share details with other officers as necessary during investigation. If you want your report kept more confidential, you can tell the Safer Dances Coordinators or other officer the confidentiality you request, and they will respect that.

We will be happy to help you contact MIT officials, law enforcement, or local support services; provide escorts; or otherwise assist you in feeling safe. We value your participation.

\section{Resources}
\label{sec:resources}

MIT's Title IX office has a list of \href{http://titleix.mit.edu/resources}{resources} that can be helpful for getting MIT to take disciplinary action or getting counseling. They also publish details on how \href{http://titleix.mit.edu/reporting}{reporting} works. The \href{http://police.mit.edu/report-crime-or-accident}{MIT Police} can be reached at 617-253-1212.

\end{document}
